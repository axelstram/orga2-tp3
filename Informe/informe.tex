
\documentclass[10pt, a4paper]{article}
\usepackage[paper=a4paper, left=1.5cm, right=1.5cm, bottom=1.5cm, top=3.5cm]{geometry}
\usepackage[latin1]{inputenc}
\usepackage[T1]{fontenc}
\usepackage[spanish]{babel}
\usepackage{indentfirst}
\usepackage{framed}
\usepackage{fancyhdr}
\usepackage{latexsym}
\usepackage{lastpage}
\usepackage{caratula}
\usepackage[colorlinks=true, linkcolor=blue]{hyperref}
\usepackage{calc}
\usepackage{graphicx}
\usepackage{color}
\usepackage{epstopdf}
\usepackage{epsfig}
\usepackage{listings}
\lstset{language=C}
\newcommand{\f}[1]{\text{#1}}

\sloppy

\hypersetup{%
 % Para que el PDF se abra a p�gina completa.
 pdfstartview= {FitH \hypercalcbp{\paperheight-\topmargin-1in-\headheight}},
 pdfauthor={C�tedra de Algoritmos y Estructuras de Datos II - DC - UBA},
 pdfkeywords={TADs b�sicos},
 pdfsubject={Tipos abstractos de datos b�sicos}
}

\parskip=5pt % 10pt es el tama�o de fuente

% Pongo en 0 la distancia extra entre �temes.
\let\olditemize\itemize
\def\itemize{\olditemize\itemsep=0pt}

% Acomodo fancyhdr.
\pagestyle{fancy}
\thispagestyle{fancy}
\addtolength{\headheight}{1pt}
\cfoot{\thepage /\pageref{LastPage}}
\renewcommand{\footrulewidth}{0.4pt}

\date{}

\begin{document}


\materia{Organizaci\'on del Computador II}
\titulo{TP N\textdegree3}
\grupo{Grupo Pistacho/Chungo}


\integrante{Axel Straminsky}{769/11}{axelstraminsky@gmail.com}
\integrante{Lucas Romero}{440/12}{lucasrafael.romero@gmail.com}
\integrante{Matias Pizzagalli}{257/12}{matipizza@gmail.com}


\maketitle


\section{GDT y modo protegido.}

Para empezar el primer objetivo del kernel es pasar a modo protegido. Para realizar esto primero deshabilita
las interrupciones, habilita la l\'inea A20 y carga la Global Descriptor Table (GDT). El kernel carga la GDT
con la instrucci\'on " lgdt [GDT\_DESC] " donde GDT\_DESC es un descriptor en un struct declarado en el archivo
gdt.h. El descriptor contiene el tama\~no de la GDT y su direcci\'on (En ese orden).

La GDT que carga es inicializada en el archivo gdt.c. En principio se cargan 4 descriptores: 2 segmentos de datos
y 2 de c\'odigo. De esos segmentos uno de c\'odigo y uno de datos tienen privilegio 0 (kernel) y los otros dos
cuentan con privilegio 3 (usuario). Estos cuatro descriptores se encuentran en los \'indices 0x12 a 0x14 ya que
las primeras 18 entradas estan reservadas por la c\'atedra.

Los cuatro segmentos tienen base en 0x00000000 y l\'imite 0x6FFFF que corresponden con un modelo de segmentaci\'on
flat. Adem\'as, los segmentos de c\'odigo permiten la lectura y los de datos permiten la escritura. Todos los
descriptores tienen el bit Default/Big seteado en 1 de manera que los segmentos son de 32 bits y est\'an todos presentes.
Para los 4 segmentos, el bit de granularidad est\'a en 1.

Entonces, los 4 descriptores de segmento quedan seteados de esta manera:


\begin{center}
\begin{tabular}{| c || c || c || c || c |}
	\hline
		Atribute & Code Kernel & Data Kernel & Code User & Data User \\
	\hline
		limit(0:15) & 0xFFFF & 0xFFFF & 0xFFFF & 0xFFFF \\
		base(0:15) & 0x0000 & 0x0000 & 0x0000 & 0x0000 \\
		base(23:16) & 0x00 & 0x00 & 0x00 & 0x00 \\
		type & 0x0A & 0x02 & 0x0A & 0x02 \\
		s & 0x01 & 0x01 & 0x01 & 0x01 \\
		dpl & 0x00 & 0x00 & 0x03 & 0x03 \\
		p & 0x01 & 0x01 & 0x01 & 0x01 \\
		limit(16:19) & 0x06 & 0x06 & 0x06 & 0x06 \\
		avl & 0x00 & 0x00 & 0x00 & 0x00 \\
		l & 0x00 & 0x00 & 0x00 & 0x00 \\
		d/b & 0x01 & 0x01 & 0x01 & 0x01 \\
		g & 0x01 & 0x01 & 0x01 & 0x01 \\
		base(31:24) & 0x00 & 0x00 & 0x00 & 0x00\\
	\hline
	

\end{tabular}
\end{center}

Ya con la GDT inicializada y cargada podemos entrar en modo protegido. para hacer esto simplemente cargamos el
contenido del registro de control 0 (cr0) en eax, efectuamos un 'or eax, 1' y movemos ese resultado de vuelta a
cr0. De esta manera queda seteado el bit PE (Protect Enable) y empezamos a trabajar en modo protegido.

Lo siguiente ser\'ia darle a los selectores de segmento su valor correspondiente. Primero efectuamos la instrucci\'on
" jmp 0x90:mp ", (mp es un label situado exactamente luego de la instrucci\'on) con esta simple instrucci\'on saltamos
al segmento de c\'odigo de privilegio 0 y ponemos en CS el \'indice para este segmento. Para el resto de los selectores
simplemente movemos 0x98 al selector (\'indice 0x13) para que queden apuntando al segmento de datos del kernel. Luego
de realizar esto seteamos la pila del kernel con la instrucci\'on " mov esp, 0x27000 ".

Lo que se nos pide a continuaci\'on es armar otro segmento que describa la posici\'on en memoria de la pantalla que
solo pueda ser accedido por el kernel. Con este fin declaramos en el archivo gdt.c otra entrada a la gdt en el \'indice
0x15 que describa un segmento de datos de 32 bit con privilegio 0 y que sea de lectura y escritura. La base de este
nuevo segmento est\'a en la direcci\'on 0xB8000 y tiene un l\'imite de 0x7FFF. El bit de granularidad est\'a en 0
con lo cual el l\'imite se calcula normalmente.


\begin{center}
\begin{tabular}{| c || c |}
	\hline
		Atribute & Screen Area\\
	\hline
		limit(0:15) & 0x7FFF\\
		base(0:15) & 0x8000 \\
		base(23:16) & 0x0B \\
		type & 0x0A \\
		s & 0x01 \\
		dpl & 0x00\\
		p & 0x01 \\
		limit(16:19) & 0x00\\
		avl & 0x00\\
		l & 0x00\\
		d/b & 0x01\\
		g & 0x00\\
		base(31:24) & 0x00\\
	\hline

\end{tabular}
\end{center}

Con este \'ultimo segmento podemos terminar el primer ejercicio y escribir la rutina requerida. Esta rutina se
puede dividir en 2 partes. La primera, una llamada a una funci\'on en C que limpia la pantalla y la deja en
blanco, esta funci\'on se encuentra en screen.c. La segunda, son dos peque\~nos ciclos en Assembler que cada uno
se encarga de pintar la primera o la \'ultima l\'inea de la pantalla de fondo negro con caracteres blancos usando el
segmento descripto en el punto anterior.


\section{IDT y rutinas de atenci\'on de excepciones}

Ya en modo protegido el objetivo consiste en armar una Interruption Descriptor Table (IDT) y escribir las rutinas de atenci\'on,
para poder atender en principio las excepciones que puedan surgir y m\'as adelante las interrupciones de reloj, teclado y syscalls.

Para este prop\'osito hay definida una macro en C para completar las entradas de la IDT.

\begin{footnotesize}
\begin{lstlisting}

#define IDT_ENTRY(numero, privilegio)
idt[numero].offset_0_15 = (unsigned short) ((unsigned int)(&_isr ## numero) & (unsigned int) 0xFFFF);
idt[numero].segsel = (unsigned short) 0x90;
idt[numero].attr = (unsigned short) (0x8E00  | (privilegio << 13));
idt[numero].offset_16_31 = (unsigned short) ((unsigned int)(&_isr ## numero) >> 16 & (unsigned int) 0xFFFF);

\end{lstlisting}
\end{footnotesize}

De esta forma, cuando llamamos a idt\_inicializar() inicializamos las primeras 20 entradas de la IDT con los siguientes valores:

\begin{center}
\begin{tabular}{| c || c |}

	\hline
		Atributo & Valor\\
	\hline
		Selector & 0x90 (Kernel data segment)\\
		Offset & (Offset correspondiente a la isr de la excepci\'on)\\
		Tipo & 0x0E (Interrupt gate, 32-bit)\\
		DPL & 0x00\\
		Presente & 0x01\\
		Sin usar & 0x00\\
	\hline

\end{tabular}
\end{center}

Para las rutinas de atenci\'on de las excepciones escribimos una macro en assembler que se ocupa de detectar la excepci\'on,
imprimirla en pantalla y colgar la ejecuci\'on ejecutando jmp \$ .

\begin{verbatim}

%macro ISR 1
global _isr%1

_isr%1:
.loopear:
	
    cli
    pushad
	
    mov eax, interrupciones
    mov ebx, %1 
		
    mov eax, [eax + ebx * 4]
	
    push eax
    call imprimir_interrupcion
    pop eax
	
    popad
    sti
	
	
%    jmp $

%endmacro

\end{verbatim}


Donde interrupciones es un arreglo de strings que contienen los mensajes a imprimir por cada excepci\'on
e imprimir\_interrucion es una rutina que toma un string y lo imprime en la parte correspondiente al \'ultimo problema.

Para testear esto generamos excepciones (tanto intencionales como accidentales) como divide by zero, page fault,
general protection, invalid tss; entre otras. Los resultados fueron los esperados.


\section{Paginaci\'on y rutinas de pantalla}

Lo primero fue escribir las rutinas que escriben los buffer de video para la pantalla de mapa y de estado. Estas
funciones fueron hechas en C y escriben dos buffer, uno en la direcci\'on 0x2D000 y otro en la direcci\'on 0x2E000.
Estos corresponder\'an a las pantallas de estado y de mapa respectivamente. Est\'an escritas para presentar una versi\'on gen\'erica
de las figuras 9 y 10.

%Poner screenshots de la pantalla de mapa y de estado? que opinan?

Luego, para activar paginaci\'on lo \'unico que hay que hacer es setear el bit 31 del registro cr0, pero para llegar a eso
primero hay que armar las estructuras necesarias. En este caso, un directorio de p\'aginas y por lo menos una entrada
en este directorio con una tabla de p\'aginas que tenga entradas v\'alidas para mapear memoria.

Con este fin, cargamos en el registro cr3 la posici\'on que ocupar\'a el page directory del kernel (0x28000) y llamamos
a la funci\'on mmu\_inicializar\_dir\_kernel. Es una funci\'on escrita en C y que, como lo indica su nombre, se encarga de
inicializar el mapa de memoria del kernel.

Previamente a esto, utilizamos las funciones inicializar\_page\_directory e inicializar\_page\_table, las cuales ponen en 0 todos 
los atributos de las entradas del page directory y page table, salvo el atributo Read/Write que dejan en 1. Esto se hace tanto para
la p\'agina del page directory como para las 2 page tables. Ya inicializadas las tablas, se las modifica para que
realicen identity mapping, las 1024 entradas de la primera tabla (los primeros 4 MB) y las primeras 896 entradas de la segunda
tabla (los 3,5 MB restantes). Una vez que tienen seteados los valores, son finalmente agregadas al page directory. Las primeras
256 p\'aginas (1 MB) las mapea con privilegio 0 y el resto con privilegio 3

Finalmente el directorio, y tablas quedan con los siguientes valores:

\begin{center}
\begin{tabular}{| c || c | c | c | c | c | c | c | c | c |}

	\hline
		PDE & Base & G & PS & A & PCD & PWT & U/S & R/W & P\\
	\hline
		0 & (Base de la tabla) & 0b & 0b & 0b & 0b & 0b & 0b & 1b & 1b \\
		1 & (Base de la tabla) & 0b & 0b & 0b & 0b & 0b & 0b & 1b & 1b \\
		2 a  1023 & 0x00 & 0b & 0b & 0b & 0b & 0b & 0b & 1b & 0b\\
	\hline

\end{tabular}


\begin{tabular}{| c | c || c | c | c | c | c | c | c | c | c |}

	\hline
		Tabla & PTE & Base & G & PS & A & PCD & PWT & U/S & R/W & P\\
	\hline
		0 & (primeras 256 n) & n & 0b & 0b & 0b & 0b & 0b & 0b & 1b & 1b \\
		0 & (restantes 768 n) & n & 0b & 0b & 0b & 0b & 0b & 1b & 1b & 1b \\
		1 & (primeras 896 n) & n + 1024 & 0b & 0b & 0b & 0b & 0b & 1b & 1b & 1b \\
		1 & (restantes)& 0x00 & 0b & 0b & 0b & 0b & 0b & 0b & 1b & 0b\\
	\hline

\end{tabular}
\end{center}
\section{Memory Management Unit}

Para inicializar el mapa de memoria de las tareas, usamos dos funciones: inicializar\_tareas y mmu\_inicializar\_dir\_tareas. El objetivo de la primera es
inicializar los directorios y tablas de p\'aginas de las tareas. Primero inicializa todos los valores en 0, salvando el bit de read/write, luego 
mapea los primeros 7,5 MB de p\'aginas con identity mapping de nivel 0 y agrega esas 2 p\'aginas en las primeras 2 posiciones del page directory. 
Finalmente termina agregando las 3 p\'aginas de la tarea en la posici\'on 0x100 del directorio de p\'aginas.

Cada tarea cuenta con 6 p\'aginas: 1 para el page directory, 3 para las page tables, 1 para la pila de c\'odigo, y otra para la pila de la bandera. Estas p\'aginas 
se encuentran a partir de la posicion 0x31000.
La funci\'on mmu\_inicializar\_dir\_tareas se encarga de mapear dos p\'aginas en el mar y una p\'agina en la tierra en el mapa de memoria de cada tarea.
Luego de esto copia las dos p\'aginas de c\'odigo de la tarea y la bandera en el area de mar y setea los atributos relacionados con la memoria para cada
tarea. Finalmente, inicializa el arreglo global direcciones\_tareas, el cual guarda las direcciones f\'isicas actuales de las 3 p\'aginas
de cada tarea, informaci\'on que utilizamos en las funciones de pantalla.

Al terminar la inicializaci\'on de la mmu cada tarea mapea  las siguientes direcciones:

\begin{center}
\begin{tabular}{| c | c |}

	\hline
		Direcci\'on virtual & Direcci\'on f\'isica \\
	\hline
		0x00000000-0x0077ffff & 0x00000000-0x0077ffff  \\
		0x40000000-0x40001fff & 0x100000 + (0x2000 * id\_tarea) - 0x101fff + (0x2000 * id\_tarea) \\
		0x40002000-0x40002fff & 0x00000000-0x00000fff \\
	\hline

\end{tabular}
\end{center}

Para realizar las dos funciones descriptas mas arriba nos proponian implementar y utilizar dos funciones mmu\_mapear\_pagina y mmu\_unmapear\_pagina, pero por decisiones
 nuestras, decidimos implementar la primera tal cual proponen pero la segunda retocarla para darle un mejor uso. Ambas funciones luego nos resultan utiles para otras
  aplicaciones como en los syscalls de navegar y fondear.
 
%~ La primera de ellas tiene la siguiente aridad mmu\_mapear\_pagina(unsigned int virtual, unsigned int cr3, unsigned int fisica, unsigned int attr) que lo que hace es:
 %~ Dividir la direccion virtual en sus tres partes, los 12 bits de offset, los 10 bits de la page table y los 10 bits de la page directory.
 %~ Luego con el cr3 (que contiene la direccion base del directorio) que nos pasan por parametro buscamos la entrada del directorio correspondiente a los 10 bits de la
 %~ page directory que obtuvimos antes. De esa entrada del directorio obtenemos la direccion base de la tabla de paginas y junto con los 10 bits de la page table que
 %~ obtuvimos mas arriba conseguimos la entrada de la pagina de tablas donde debemos colocar los datos correspondientes para mapear la direccion fisica que nos pasan 
 %~ por parametros con los atributos que nos pasan por parametro.
 
Lo que hace la primera de ellas basicamente es obtener la informacion sobre la page table y la page directory de la direccion virtual pasada por parametro, para acceder
 a las entradas en el directorio y tabla de paginas correspondiente. para mapear ahi la direccion fisica que nos pasan por parametro con los atributos tambien pasados por
 parametro.
 
\begin{footnotesize}
\begin{lstlisting}

void mmu_mapear_pagina(unsigned int virtual, unsigned int cr3, unsigned int fisica, unsigned int attr) {
	
	page_directory_entry* page_directory = (page_directory_entry*) cr3;
	virtual = virtual >> 12; //borro el offset
	unsigned int pte = virtual & 0x3ff;
	virtual = virtual >> 10; //borro el table
	unsigned int pde = virtual;
	
	unsigned int page_table_address = (page_directory[pde].page_table_address << 12);
	
	page_table_entry* page_table = (page_table_entry*) page_table_address;

	page_table[pte].present = attr & 0x1;
	page_table[pte].read_write = (attr & 0x2) >> 1;
	page_table[pte].user_supervisor = (attr & 0x4) >> 2;
	page_table[pte].physical_page_address = (fisica >> 12);

}

\end{lstlisting}
\end{footnotesize}

La segunda funcion, mmu\_unmapear\_pagina, decidimos implementarla diferente para poder utilizarla de mejor manera. Lo que decidimos es que la funcion en vez de unmapear la direccion
 virtual pasada por parametro, que solamente nos devuelva en que direccion fisica se encuentra mapeada, para luego nosotros encargarnos de pisar esa misma direccion con la funcion
 descripta mas arriba.
 
La implementacion de la funcion basicamente lo que hace es descomponer la direccion virtual en sus tres componentes para luego utilizar esas componentes para acceder a las entradas
 en el directorio y tabla de paginas correspondientes y obtener la direccion fisica a donde se encuentra mapeada esa direccion virtual.

\begin{footnotesize}
\begin{lstlisting}

unsigned int mmu_unmapear_pagina(unsigned int virtual, unsigned int cr3) {

	page_directory_entry* page_directory = (page_directory_entry*) cr3;
	virtual = virtual >> 12; //borro el offset
	unsigned int pte = virtual & 0x3ff;
	virtual = virtual >> 10; //borro el table
	unsigned int pde = virtual;

	page_table_entry* page_table = (page_table_entry*) (page_directory[pde].page_table_address << 12);
	unsigned int fisica = (page_table[pte].physical_page_address << 12);

	return fisica;

}
\end{lstlisting}
\end{footnotesize}


Adicionalmente, aprovechamos para obtener los offsets de las funciones bandera. Para obtener estos offsets de la direcci\'on virtual 0x40001FFC de cada tarea, lo que hicimos fue lo siguiente: como la direcci\'on virtual
0x40001FFC hace referencia a la segunda p\'agina f\'isica de cada tarea, con un offset de 0x1FFC, creamos un arreglo global llamado offsets\_funciones\_bandera, 
en donde recopilamos esta informaci\'on. Para esto, directamente accedimos a las posiciones 0x10000, 0x12000, etc.,
con un offset de 0x1FFC. Luego, a la hora de inicializar o reinicializar las TSSs de las banderas, utilizamos directamente los offsets de este arreglo, a los cuales les sumamos la direcci\'on virtual 0x40000000, valor que 
utilizamos como EIP de las mismas.



\section{Interrupciones de hardware y syscalls}

Al igual que para las excepciones, usamos el define en el el archivo idt.c para inicializar las entradas en la IDT que corresponder\'an a las interrupciones
de reloj (Int 0x20), teclado (Int 0x21) y syscalls (Int 0x50 e Int 0x66). Las entradas de la IDT quedan con la siguiente informaci\'on:

\begin{center}
\begin{tabular}{| c || c | c |}

	\hline
		Atributo & Int 0x20 y 0x21 & Int 0x50 y 0x66\\
	\hline
		Selector & 0x90 (Kernel data segment) & 0x90 (Kernel data segment)\\
		Offset & (Offset correspondiente a la isr de la interrupci\'on) & (Offset correspondiente a la isr de la interrupci\'on)\\
		Tipo & 0x0E (Interrupt gate, 32-bit) & 0x0E (Interrupt gate, 32-bit)\\
		DPL & 0x00 & 0x03\\
		Presente & 0x01 & 0x01\\
		Sin usar & 0x00 & 0x00\\
	\hline
\end{tabular}
\end{center}

Con las entradas de la IDT ya definidas, el paso siguiente es escribir las rutinas de atenci\'on. En esta etapa, solo se le dan funcionalidades b\'asicas,
sin embargo posteriormente las interrupciones ser\'an las que realicen el trabajo de scheduling.

Inicialmente, la interrupci\'on de reloj solo se encarga de llamar a la funci\'on proximo\_reloj. La interrupci\'on de teclado muestra los buffer de video
de mapa y estado e imprime un n\'umero en la esquina superior derecha de la pantalla. Las interrupciones 0x50 y 0x66 cambian el valor de eax por 0x42

Estas son las rutinas:

\begin{verbatim}
_isr32:
    cli
    pushad

    call fin_intr_pic1

    call proximo_reloj

    popad
    sti
    iret

_isr33:

    cli
    pushad
    call fin_intr_pic1
	
    xor eax, eax
    in al, puerto_teclado
	
    push eax
    call interrupcion_teclado
    add esp, 4
	
;Interrupcion_teclado es una funcion en C que imprime en pantalla el numero
;presionado o que cambia los buffer a mapa o estado segun corresponda

    popad
    sti
    iret

;isr 0x50
_isr80:

    mov eax, 0x42
	iret

;isr 0x66
_isr102:

    mov eax, 0x42
	iret

\end{verbatim}



\section{TSS}

Lo primero que hicimos fue definir las entradas de la GDT que corresponder\'ian a los descriptores de las TSS. Estas ocupan desde el \'indice
23 al 40. Son 2 por cada tarea (nav\'io y bandera), una para la tarea idle y una para la tarea inicial. Todas son cargadas con la misma informaci\'on.
El atributo base es luego cargado din\'amicamente al inicializar las tss. Las entradas de la GDT quedan con la siguiente informaci\'on para todas las tareas.

\begin{center}
\begin{tabular}{| c || c |}
	\hline
		Atributo & Valor\\
	\hline
		limit(0:15) & 0x0067\\
		base(0:15) & 0x0000 \\
		base(23:16) & 0x00 \\
		type & 0x09 \\
		s & 0x00 \\
		dpl & 0x00\\
		p & 0x01 \\
		limit(16:19) & 0x00\\
		avl & 0x00\\
		l & 0x00\\
		d/b & 0x00\\
		g & 0x00\\
		base(31:24) & 0x00\\
	\hline

\end{tabular}
\end{center}

Con las entradas de la GDT inicializadas, el paso siguiente es setear los valores del tss para cada tarea, incluyendo las tareas idle e inicial.
El enuniciado dicta que valores deben llevar las tss de las tareas idle, nav\'io y bandera. En el caso de la tss inicial solo hay que poner una tss valida
en la gdt pues solo se usa para dar el primer salto. Para inicializar la tarea idle, llamamos a la funci\'on tss\_inicilizar\_tarea\_idle

\begin{center}
\begin{tabular}{| c || c |}

	\hline
		Atributo & idle \\
	\hline
		cs & 0x90 \\
		ds, es, fs, gs, ss & 0x98 \\
		registros de prop\'osito general & 0x00 \\
		ebp, esp & 0x40001C00 \\
		eflags & 0x202 \\
		eip & 0x40000000 \\
		esp0 & 0x2A000 \\
		ss0 & 0x98 \\
		cr3 & 0x28000 \\
	\hline

\end{tabular}
\end{center}

Una de las particularidades del procesador es que permite una conmutaci\'on de tareas autom\'atica, facilitando bastante el trabajo. Usando la instrucci\'on
\verb| jmp selector: offset |, donde selector es un selector de segmento que apunte a una entrada valida de tss en la gdt, el procesador ejecuta un task switch
hacia esa tarea sin la necesidad de trabajo adicional.


\section{Scheduler}

Para inicializar los arreglos de tss de las tareas usamos la funcion tss\_inicializar. Es una funci\'on en el archivo tss.c que se ocupa de inicializar las tss
de las tareas y las banderas. Los valores son los provistos por la c\'atedra y quedan de la siguiente manera:

\begin{center}
\begin{tabular}{| c || c | c |}

	\hline
		Atributo & nav\'io & bandera \\
	\hline
		cs & 0xA3 & 0xA3 \\
		ds, es, fs, gs, ss & 0xAB & 0xAB \\
		registros de prop\'osito general & 0x00 & 0x00 \\
		ebp, esp & 0x40001C00 & 0x40001FFC \\
		eflags & 0x202 & 0x202 \\
		eip & 0x40000000 & [0x40001FFC] + 0x40000000 \\
		esp0 & 0x35000 + (0x6000 * id\_tarea) & 0x36000 + (0x6000 * id\_tarea) \\
		ss0 & 0x98 & 0x98 \\
		cr3 & 0x31000 + (0x6000 * id\_tarea) & 0x31000 + (0x6000 * id\_tarea) \\
	\hline

\end{tabular}
\end{center}

Donde id\_tarea es el n\'umero de tarea a la cual se hace referencia.Los campos no mencionados tienen valor 0x00.

El paso siguiente es implementar las funciones sched\_proximo\_indice() y sched\_proxima\_bandera. Estas dos funciones las implementamos en C en el archivo sched.c.
Y las escribimos as\'i:

\begin{lstlisting}
unsigned short sched_proximo_indice() {
	
	unsigned short i = (ultima_tarea_ejecutada) % 8;
       
    while (tareas_corriendo[i] == 0) {
               
    	i = (i + 1) % 8; //Si no hay tareas corriendo, se cuelga.
               
    }
       
    return tareas_corriendo[i];
	
}
\end{lstlisting}

ultima\_tarea\_ejecutada es una variable global sobre la que se escribe el id\_tarea de la \'ultima tarea que se ejecut\'o y el array tareas\_corriendo es un array que
tiene en cada \'indice el \'indice de la gdt correspondiente a la tarea si la tarea est\'a corriendo o 0x00 si la tarea fue desalojada.

\begin{lstlisting}
short sched_proxima_bandera() {

	static unsigned short i = 0;
	while(i < 8) {

		if (banderas_corriendo[i] != 0) {
			i++;
			return banderas_corriendo[i-1]; 
	
		} else {
	
			i++;
		}
	}
	
	i = 0;
	return -1;
	
}

\end{lstlisting}

Donde el array banderas\_corriendo es un array que tiene en cada \'indice el \'indice de la gdt correspondiente a la bandera si la tarea est\'a corriendo o 0x00 si la tarea fue desalojada.

Con esas dos funciones hechas, lo siguiente que pide el enunciado es modificar la int 0x50 para que responda a los syscalls seg\'un corresponda. Lo primero que hace la int 0x50 es desactivar
las interrupciones y chequear que quien la llam\'o no haya sido una bandera. Si ese fuera el caso, desaloja la tarea y la bandera, e imprime en pantalla el \'ultimo problema. Si la llam\'o una
tarea compara el par\'ametro que se le pasa por eax y decide que servicio llamar: game\_fondear, game\_canonear o game\_navegar. Estas son 3 funciones escritas en C en el archivo game.c que
realizan el servicio en s\'i. Si el parametro en eax no coincide con ning\'un c\'odigo de servicio, salta al final de la interrupci\'on. Al final la int 0x50 vuelve a setear las interrupciones
y salta a la tarea idle.

A continuaci\'on modificamos la int 0x20 (reloj) para que en cada tick haga la conmutaci\'on de tareas. Primero chequea que no se est\'e ejecutando una bandera, de ser as\'i desaloja la tarea 
y su bandera. Despu\'es chequea que la pr\'oxima tarea a saltar no sea la misma que se est\'a ejecutando, si es as\'i no salta y simplemente hace un iret. Si el \'indice al cu\'al saltar no
corresponde a la tarea actual, escribe el selector en una variable local y hace un jmp far usando esa variable.

Una vez hecho esto, agregamos un contador: contador\_bandera en 0. Cada vez que se va a saltar a una tarea, se aumenta en 1 el contador. Si el contador est\'a en 3 empieza a llamar, una por una
las banderas h\'abiles usando la funci\'on sched\_proxima\_bandera. Una vez que llam\'o a la \'ultima vuelve a setear el contador en 0 para que la pr\'oxima interrupci\'on de reloj llame a la tarea
que corresponda.

Finalmente, lo \'unico que queda es modificar las isr de las excepciones para que impriman en pantalla el \'ultimo problema, desalojen la tarea que las produjo y salten a la idle.
Para hacer esto lo primero que chequea la excepci\'on es si tiene un error code o no, porque esto cambia la forma de pushear par\'ametros para la funci\'on imprimir\_ultimo\_problema, una funci\'on
escrita en C que se encuentra en el archivo screen.c. Ya llamada la funci\'on chequea si la produjo una bandera o una tarea, y desaloja ambas de acuerdo al resultado. Terminado esto hace un jmp a la
tarea idle. 


\end{document}
